\documentclass[UTF8,a4paper,sub3section]{ctexart}
\usepackage{amsmath,longtable,fancyhdr}
\usepackage{indentfirst,adjustbox}
\usepackage{etoolbox}
\usepackage{xcolor}
\usepackage{numprint}

\usepackage{lastpage,caption,graphics}
\usepackage{titlesec}
%\usepackage{hyperref}
\usepackage{ltxtable,tabularx,booktabs,threeparttable,graphicx }
\usepackage{titlesec,float,rotating}
\setcounter{secnumdepth}{4}
%\titleformat{\chapter}[display]{\centering\Huge\bfseries}{第\,\thechapter,章}{1em}{}

%\renewcommand\chaptername{\arabic{chapter}}
\let\stdsection\section
\let\stdtableofcontents\tableofcontents
%\let\stdchapter\chapter
%\renewcommand\thechapter{\arabic{chapter}}
%\renewcommand\chapter{\newpage \pagestyle{fancy} \stdchapter}
%\renewcommand\tableofcontents{\pagestyle{empty}  \stdtableofcontents}
\renewcommand\section{\newpage\stdsection}
%\renewcommand{\cleardoublepage}{}
%\renewcommand{\clearpage}{}
\newcommand{\tabincell}[2]{\begin{tabular}{@{}#1@{}}#2\end{tabular}}



\CTEXsetup[name={,}]{section}
\CTEXsetup[number=\arabic{section}]{section}
%\CTEXsetup[number=\arabic{section}]{section}
%\CTEXsetup[format=\raggedright\bfseries\zihao{1}]{chapter}
\CTEXsetup[beforeskip=0pt,afterskip=0pt]{section}
\CTEXsetup[format=\raggedright\bfseries\zihao{-4}]{section}

%\CTEXsetup[number=\arabic{section}.\arabic{subsection}]{subsection}
\captionsetup{labelsep=period}


\setlength{\parindent}{2em}
%\rhead{   \textcolor{blue}{北京赛默百合生物科技有限公司}  }
%\lhead{\rightmark}



\fancyhead[OR]{ \zihao{5}\textcolor{blue}{北京赛默百合生物科技有限公司}}

\fancyhead[OL]{ the page  \thepage   of  \pageref{LastPage} }



\input{"Title.tex"}
\begin{document}

\maketitle
\thispagestyle{empty}
\newpage

\pagestyle{plain}
\setcounter{page}{1}
\pagenumbering{Roman}
\tableofcontents
\newpage

\setcounter{page}{1}
\pagenumbering{arabic}
\pagestyle{fancy}
\input{"QC.tex"}
\section{基础分析}
\subsection{序列组装}
使用HGAP\cite{HGAP}对获得的测序数据进行组装。
\subsubsection{组装方法}
由于Pacbio的测序数据质量较低,并且其错误呈现高随机性,因此,可以使用短的reads比对到长的reads上,并通过投票的方法对每一个碱基进行清洗,从而得到正确率能够满足分析需求的数据\cite{HGAP}。


而后,使用经典的OLC算法架构的组装工具将经过算法处理后的reads进行组装,得到最终的组装结果。


\input{"Assembly_Stat.tex"}
\input{"RepeatMasker.tex"}

\input{"GenePrediction.tex"}
\input{"Annotation.tex"}
\input{"IS.tex"}
\input{"Cripsr.tex"}
\input{"Prophage.tex"}

\input{"Genomic_Island.tex"}

\input{"{{RootPath}}Appendix.tex"}



\bibliographystyle{unsrt}

\bibliography{{{RootPath}}Bacteria}




\end{document}
